%%%%%%%%%%%%%%%%%%%%%%%%%%%%%%%%%%%%%%%%%%%%%%%%%%%%%%%%%%%%%%%%%%%%%%
% LaTeX Template: Two Column Colour Article
%
% Source: http://www.howtotex.com/
% Feel free to distribute this template, but please keep the
% referal to howtotex.com.
% Date: Feb 2011
%%%%%%%%%%%%%%%%%%%%%%%%%%%%%%%%%%%%%%%%%%%%%%%%%%%%%%%%%%%%%%%%%%%%%%
% adaptions made by wolfgang stoettner mail@stoettner.net
%%%%%%%%%%%%%%%%%%%%%%%%%%%%%%%%%%%%%%%%%%%%%%%%%%%%%%%%%%%%%%%%%%%%%%

%%% Preamble
\documentclass[	DIV=calc,%
							paper=a4,%
							fontsize=10pt,%
							twocolumn]{scrartcl} % KOMA-article class
\usepackage[french]{babel}	% English language/hyphenation
\usepackage[protrusion=true,expansion=true]{microtype}	% Better typography
\usepackage{amsmath,amsfonts,amsthm} % Math packages
\usepackage{pythontex} % Math packages
\usepackage[pdftex]{graphicx} % Enable pdflatex
\usepackage{wrapfig} % enable figure wrapping
\usepackage[svgnames]{xcolor} % Enabling colors by their 'svgnames'
\usepackage[hang, small,labelfont=bf,up,textfont=it,up]{caption} % Custom captions under/above floats
\usepackage{epstopdf} % Converts .eps to .pdf
\usepackage{subfig}	% Subfigures
\usepackage{booktabs} % Nicer tables
\usepackage{fix-cm}	% Custom fontsizes
\usepackage{booktabs} % prof. looking tables (www.en.wikibooks.org/wiki/LaTeX/Tables#Professional_tables)
\usepackage{float}
\usepackage{tgtermes}
\usepackage[T1]{fontenc}
\usepackage[utf8]{inputenc}
\usepackage{pdfpages}
\usepackage[smartEllipses]{markdown}

%%% Custom sectioning (sectsty package)
\usepackage{sectsty} % Custom sectioning (see below)
\allsectionsfont{%		% Change font of al section commands
	\usefont{OT1}{phv}{b}{n}%	% bch-b-n: CharterBT-Bold font
	}

\sectionfont{%		% Change font of \section command
	\usefont{OT1}{phv}{b}{n}%	% bch-b-n: CharterBT-Bold font
	}
%%% Headers and footers
\usepackage{fancyhdr} % Needed to define custom headers/footers
	\pagestyle{fancy} % Enabling the custom headers/footers
\usepackage{lastpage}	

% Header (empty)
\lhead{}
\chead{}
\rhead{}
% Footer (you may change this to your own needs)
\lfoot{\footnotesize \texttt{formulaire mécatronique 2} \textbullet \vspace{5pt} Antonin Kenzi}
\cfoot{}
\rfoot{\footnotesize page \thepage\ of \pageref{LastPage}}	% "Page 1 of 2"
\renewcommand{\headrulewidth}{0.0pt}
\renewcommand{\footrulewidth}{0.4pt}
\newcommand{\hformbar}[1]{\bigskip\hrule\vspace{5pt}} % creates a horizontal bar to separate formulae better; space adaptions can be made centrally here


%%% Creating an initial of the very first character of the content
\usepackage{lettrine}
\newcommand{\initial}[1]{%
     \lettrine[lines=3,lhang=0.3,nindent=0em]{
     				\color{DarkGoldenrod}
     				{\textsf{#1}}}{}}

%%% Title, author and date metadata
\usepackage{titling} % For custom titles

\newcommand{\HorRule}{\color{DarkGoldenrod}%	% Creating a horizontal rule
									  	\rule{\linewidth}{1pt}%
										}
\pretitle{\vspace{-83pt} \begin{flushleft} \HorRule 
				\fontsize{15}{15} \usefont{OT1}{phv}{b}{n} \color{DarkRed} \selectfont 
				}
\title{Formulaire Reg Num} % Title of your article goes here
\posttitle{\par\end{flushleft}}
\preauthor{\vspace{-20pt} \begin{flushleft}\large \usefont{OT1}{phv}{b}{sl} \color{DarkRed}}
\author{Kenzi Antonin}	% Author name goes here
\postauthor{\vspace{-20pt} \footnotesize \usefont{OT1}{phv}{m}{sl} \color{Black}  \par\end{flushleft}\HorRule}
\date{\vspace{-30pt} \today} % No date
\newcounter{mycounter}
%%% wws: create a non-indented formula name
\newcommand{\formdesc}[1]{\noindent\textbf{#1} \addtocounter{mycounter}{1} \hfill \themycounter}

\newcommand{\formtitle}[1]{\noindent\underline{#1}}

%%% Begin document -----------------------------------------------------------------
\begin{document}
{\maketitle }
\thispagestyle{fancy} 	% Enabling the custom headers/footers for the first page 
% The first character should be within \initial{}

\formdesc{Moteur à courant continu}
\begin{enumerate}
    \item Stator : fixe
    \item Rotor : mobile
    \item Collecteur : élément d'usure 
\end{enumerate}

\formtitle{Production de couple :}
\begin{itemize}
    \item spire parcourue par un courant $i_a(t)$
    \item Champ d'induction B 
\end{itemize}
Par Laplace :

{\hfill $\vec{F} = (\vec{i_a}(t) \wedge  \vec{B})\cdot l$\hfill}

cette force produit un couple faisant tourner le moteur : 

{\hfill $T_{em}(t) = k_T \cdot i_a(z)$\hfill}

cela induit une tension :

{\hfill $u_i(t) = \cfrac{d \Phi(t)}{dt}=k_u \cdot \omega(t)$\hfill}

\hformbar

\formdesc{Équations du moteur Dc et de la charge}
{ \vspace{-3mm}
    \begin{figure}[H]
        \includegraphics[width=0.49\textwidth]{img/equation_mot.JPG}
    \end{figure}
}
\formtitle{Équation de tension : }

{\hfill $u_a(t) = R_a \cdot i_a + L_a \cfrac{di_a(t)}{dt}+u_{im}(t)$\hfill}

{\hfill $I_a(s) = [U_a(s).U_{im}(s)] \cdot \cfrac{ \cfrac{1}{R_a}}{1+s\cfrac{L_a}{R_a}}  $\hfill}


\formtitle{Équation de mouvement :} 

{\hfill $Jt \cfrac{d\Omega(t)}{dt} = T_{em}(t) - C_v \Omega(t) - Tres(t) $\hfill}

{\hfill $\Omega(s) = [T_{em}(s).T_{res}(s)] \cdot \cfrac{ \cfrac{1}{C_v}}{1+s\cfrac{J_t}{C_v}}  $\hfill}

\formtitle{Équation de couplage : }

{\hfill $k_T = \cfrac{T_{em}(t)}{i_a(t)} = \cfrac{u_{im}(t)}{\Omega(t) }= k_u$\hfill}

\formtitle{Hypothèses :}
\begin{enumerate}
    \item Linéarité : pas de saturation du fer, résistance Cst
    \item Frottements négligeables, généralement < 5\% du couple nominal
    \item arbre rigide en torsion : ce n’est pas toujours le cas
\end{enumerate}

\formdesc{Calcul des fonctions de transfert}

on introduit :
\begin{itemize}
    \item $\tau_e = \cfrac{L_a}{R_a}$
    \item $\tau_m = \cfrac{J_t\cdot R_a}{k_T\cdot k_E}$
\end{itemize}

Les fonctions de transfert du moteur DC et de la charge, sans frottements visqueux, deviennent :


{\hfill $G_{u\omega} = \cfrac{\Omega(s)}{U_a(s)} = \cfrac{1}{k_E}\cdot\cfrac{1}{1 + s \cdot \tau_m + s^2 \cdot \tau_m \cdot \tau_e} $\hfill}

\vspace{3mm}
{\hfill $G_{ui} = \cfrac{I_a(s)}{U_a(s)} = \cfrac{J_t}{k_E\cdot k_T}\cdot\cfrac{1}{1 + s \cdot \tau_m + s^2 \cdot \tau_m \cdot \tau_e} $\hfill}

\vspace{3mm}

\formtitle{Cas particulier : $\tau_m \leqslant 4 \cdot\tau_e$}

Dénominateur défini par 2 pôles réels :

\vspace{3mm}
{\hfill $s_{1,2} = \cfrac{-\tau_m \pm \sqrt{\tau_m^2 - 4 \cdot \tau_m \cdot \tau_e}}{2 \cdot \tau_m \cdot \tau_e} $\hfill}

\vspace{3mm}
ce qui décompose le système tel que : 

{\hfill $ \cfrac{1}{(1+s\cdot \tau_{max})(1+s\cdot \tau_{min})}$\hfill}
avec : {\hfill $\tau_{max,min} = -\cfrac{1}{s_{1,2}}$\hfill}

\vspace{3mm}
\formtitle{Cas particulier : $\tau_m \ggg \tau_e $}
{\hfill $ \cfrac{1}{(1+s\cdot \tau_{m})(1+s\cdot \tau_{e})}$\hfill}


\formtitle{Fonction page 1.22 :}

\hformbar


\formdesc{\color{DarkRed} Alimentations des moteurs DC}

\formtitle{Méthodes :}

\begin{enumerate}
    \item Analogique : Grande perte à basse vitesse. 
    \item Alimentation à découpage : pertes variateur constant
\end{enumerate}

\formtitle{Pont en « H » :}

\begin{figure}[H]
    \begin{center}      
        \includegraphics[width = 0.3\textwidth]{img/Pont_H.JPG}
    \end{center}
\end{figure}
\formtitle{Gestion par pulse width modulation}

\begin{enumerate}
    \item \formtitle{Par régime stationnaire : }

{\hfill $ \bar{U_{M}} = U_A * \biggl(2 \cdot \cfrac{t_e}{T_P} -1 \biggr)$\hfill}

\item \formtitle{Par «sous-oscillation» : }

\item \formtitle{Par fonction de transfert : }

    \begin{figure}[H]
        \begin{center}      
            \includegraphics[width = 0.4\textwidth]{img/PWM_FT.JPG}
        \end{center}
    \end{figure}

    {\hfill $ K_{cm} = \cfrac{U_A}{\hat{u_{cm}}}$\hfill}

\end{enumerate}

\formtitle{Approximation du retard pur : }
\begin{itemize}
    \item Si analogique 
        \begin{itemize}
            \item uh triangulaire : $T_{cm} = \cfrac{T_p}{3} $
            \item uh dent de scie : $T_{cm} = \cfrac{T_p}{2} $
        \end{itemize}
    \item Si Numérique : $T_{cm} = T_p $
\end{itemize}



\begin{table}
    \begin{tabular}{c | c}
        1er ordre : & Padé : \\
        {\hfill $ e^{-s\cdot T_{cm}} \backsimeq \cfrac{1}{1 + s\cdot T_{cm}} $\hfill} & {\hfill $ e^{-s\cdot T_{cm}} \backsimeq \cfrac{1 - s \cdot \cfrac{T_{cm}}{2}}{1 + s \cdot \cfrac{T_{cm}}{2}} $\hfill}
    \end{tabular}
\end{table}


\hformbar


\formdesc{Rappel sur le moment d’inertie}

\begin{figure}[H]
    \begin{center}      
        \includegraphics[width = 0.49\textwidth]{img/Rappel_inertie1.JPG}
    \end{center}
\end{figure}

\begin{enumerate}
    \item Courroie ou crémaillère : $J_{eq} = m \cdot r^2$
    \item Vis-à-billes : $J_{eq} = m \cdot \biggl(\cfrac{p}{2\pi}\biggr)^2$
    \item Réducteurs : $J_{eq} = J_{ch}\cdot \cfrac{1}{i^2}$
\end{enumerate}
\hformbar


\formdesc{Réglage de la tension / vitesse}

\hformbar

\formdesc{Anti-windup}


\hformbar


\formdesc{Réglage du courant / couple}
\begin{figure}[H]
    \begin{center}
        \includegraphics[width = 0.49\textwidth]{img/Regulation_courrant.JPG}

        \includegraphics[width = 0.4\textwidth]{img/Regulation_courrant_Gai.JPG}

    \end{center}
\end{figure}

{\scriptsize $G_{ai}(s) = \cfrac{K_{cm} \cdot K_{mi} \cdot J_t}{k_t \cdot k_E} \cdot \cfrac{s}{1+s \cdot \tau_m + s^2 \cdot \tau_e \cdot \tau_m} \cdot \cfrac{1}{1+s \cdot \tau_{cm}}\cdot \cfrac{1}{1+s \cdot \tau_{mi}} $\hfill}

\hformbar


\end{document}
